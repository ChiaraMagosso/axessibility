% 
% Copyright (C) 2018, 2019, 2020 by 
% Anna Capietto, Sandro Coriasco, Boris Doubrov, Alexander Koslovski,
% Tiziana Armano, Nadir Murru, Dragan Ahmetovic, Cristian Bernareggi
%
% Based on accsupp and tagpdf
%
% This work consists of the main source files axessibility.dtx and axessibility.lua,
% and the derived files
%   axessibility.ins, axessibility.sty, axessibility.pdf, README,
%   axessibilityExampleSingleLineT.tex, axessibilityExampleSingleLineA.tex,
%.  axessibilityExampleAlignT.tex, axessibilityExampleAlignA.tex
% 
% This work may be distributed and/or modified under the 
% conditions of the LaTeX Project Public License, either version 1.3
% of this license or (at your option) any later version.
% The latest version of this license is in
%   http://www.latex-project.org/lppl.txt
% and version 1.3 or later is part of all distributions of LaTeX
% version 2005/12/01 or later.
%
% This work has the LPPL maintenance status `maintained'.
% 
% The Current Maintainer of this work is 
%               Sandro Coriasco
%

\documentclass[a4paper,11pt]{article}

\usepackage{axessibility}

\title{The golden mean}
\author{}
\date{}

\begin{document}

\maketitle

\tagstructbegin{tag=P}
  \tagmcbegin{tag=P}
The golden mean is the number
  \tagmcend
\tagstructend
\[\frac{1 + \sqrt{5}}{2},\] 
\tagstructbegin{tag=P}
  \tagmcbegin{tag=P}
that is, the root larger in modulus of
  \tagmcend
\tagstructend
\begin{equation} x^2 - x - 1. \end{equation}
\tagstructbegin{tag=P}
  \tagmcbegin{tag=P}
It is usually defined as the ratio of two lengths \auxiliaryspace
  \tagmcend
\tagstructend
 \(a\)
\tagstructbegin{tag=P}
  \tagmcbegin{tag=P}
  and \auxiliaryspace
    \tagmcend
\tagstructend
\(b\) 
\tagstructbegin{tag=P}
  \tagmcbegin{tag=P}
  such that 
  \tagmcend
\tagstructend
\begin{equation*} (a+b) : a = a : b. \end{equation*} 
\tagstructbegin{tag=P}
  \tagmcbegin{tag=P}
Let \auxiliaryspace
  \tagmcend
\tagstructend
 \(x\) 
 \tagstructbegin{tag=P}
  \tagmcbegin{tag=P}
be the ratio \auxiliaryspace
  \tagmcend
\tagstructend
\( \frac{a}{b} \).
\tagstructbegin{tag=P}
  \tagmcbegin{tag=P}
We have \auxiliaryspace
  \tagmcend
\tagstructend
\( \frac{a+b}{a} = 1 + \frac{1}{x} \), 
\tagstructbegin{tag=P}
  \tagmcbegin{tag=P}
from which we get the equation \auxiliaryspace
  \tagmcend
\tagstructend
 \(x^2 = x + 1\).

\end{document}